\documentclass{article}


% load package with ``framed'' and ``numbered'' option.
\usepackage[framed,numbered,autolinebreaks,useliterate]{mcode}
\usepackage{cite}
% something NOT relevant to the usage of the package.
\setlength{\parindent}{0pt}
\setlength{\parskip}{18pt}
\title{\texttt{The Note about Histograms}}
\author{TangNing}
% //////////////////////////////////////////////////
\begin{document}
\date{}
\maketitle

\textbf{Abstract---}Histograms have been widely used in computer vision and image processing, the main uses of histogram include image enhancement, object recognition,  and image retriavel. Histogram has been basically adopted as a fundamental tool for image enhancement. But in the past few decade, histogram-based feature despritors have been receiving increasing attention due to the appearance of Histogram of Oriented Gradient (HOG). This note is divided into three parts as follows. The first part primarily summarsizes some distinct histogram-related techniques and the relation among these various techniques. The second part explains the definiton of multiresolution histogram, this part will focus in comprehending the meaning of multiresolution. And finally, the multi-dimensional histogram is discussed in the third part.
\section{Histogram-related Techniques}
\subsection{Histogram-based Image Enhancement}
One of the simplest approaches to image enhancement is based on histogram, which is usually called histogram modification or histogram transformation. The traditional methods of histogram transformation can be classified as either global\cite{Modification1978,Sapiro1997,Arici2009} or local \cite{Caselles1999}. Histogram stretch\cite{Alparslan1981}, histogram equalization\cite{Kundu1998}, and histogram matching (histogram specification)\cite{Coltuc2006,Wan2007,Mignotte2011,Sen2011} are the common methods for histogram modification. Besides, there is a least popular manner named histogram hyperbolization \cite{Nahin1979}, which could also manipulate the picture brightness levels to achieve the goal of image enhancement.\\
Histogram specification refers to a class of image transforms which aims to obtain images the histograms of which have a desired shape, and in Particular, obtaining a uniform histogram image corresponds to the well-known image enhancement technique called histogram equalization. Adaptive Histogram Equalization (local histogram equalization, or termed block-overlapped histogram equalization) is an extension of histogram equalization where the image is divided into several smaller regions and these regions are locally equalizaed to obtain more local image details. To reduce the high computation complexity of this method, sub-block nonoverlapped histogram equalization can be used. However, this nonoverlapped histogram usually produces blocking effects in output image after enhancement. Therefore, partially overlapped sub-block histogram equalization has been proposed\cite{Kim2001}. Contrast limited adaptive histogram equalization,	another extension to ordinary adaptive histogram equalization,  could limit its contrast and avoid amplified noise excessively in the image\cite{Zhu1999}. These histogram-based techniques for image enhancement could even be stretched over two-dimensional histogram or more\cite{Celik2012}.
\subsection{Histogram-based Descriptors}

 






\bibliographystyle{ieee}
\bibliography{references.bib}
\end{document}
