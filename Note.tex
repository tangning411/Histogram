\documentclass{article}


% load package with ``framed'' and ``numbered'' option.
\usepackage[framed,numbered,autolinebreaks,useliterate]{mcode}
\usepackage{cite}
% something NOT relevant to the usage of the package.
\setlength{\parindent}{0pt}
\setlength{\parskip}{18pt}
\title{\texttt{The Note about Histograms}}
\author{TangNing}
% //////////////////////////////////////////////////
\begin{document}
\date{}
\maketitle

\textbf{Abstract---}Histograms have been widely used in computer vision and image processing, the main uses of histogram include image enhancement, object recognition,  and image retriavel. Histogram has been basically adopted as a fundamental tool for image enhancement. But in the past few decade, histogram-based feature despritors have been receiving increasing attention due to the appearance of Histogram of Oriented Gradient (HOG). This note is divided into three parts as follows. The first part primarily summarsizes some distinct histogram-related techniques and the relation among these various techniques. The second part explains the definiton of multiresolution histogram, this part will focus in comprehending the meaning of multiresolution. And finally, the multi-dimensional histogram is discussed in the third part.
\section{Histogram-related Techniques}
\subsection{Histogram-based Image Enhancement}
One of the simplest approaches to image enhancement is based on histogram, which is usually called histogram modification or histogram transformation. The traditional methods of histogram transformation can be classified as either global\cite{Modification1978,Sapiro1997,Arici2009} or local \cite{Caselles1999}. Histogram stretch\cite{Alparslan1981}, histogram equalization\cite{Kundu1998}, and histogram specification (histogram matching)\cite{Coltuc2006,Wan2007,Mignotte2011,Sen2011} are the common methods for histogram modification. Besides, there is a least popular manner named histogram hyperbolization \cite{Nahin1979}, which could also manipulate the picture brightness levels to achieve the goal of image enhancement.\\
Histogram specification refers to a class of image transforms which aims to obtain images the histograms of which have a desired shape, and in Particular, obtaining a uniform histogram image corresponds to the well-known image enhancement technique called histogram equalization. Adaptive Histogram Equalization (local histogram equalization, or termed block-overlapped histogram equalization) is an extension of histogram equalization where the image is divided into several smaller regions and these regions are locally equalizaed to obtain more local image details. To reduce the high computation complexity of this method, sub-block nonoverlapped histogram equalization can be used. However, this nonoverlapped histogram usually produces blocking effects in output image after enhancement. Therefore, partially overlapped sub-block histogram equalization has been proposed\cite{Kim2001}. Contrast limited adaptive histogram equalization,	another extension to ordinary adaptive histogram equalization,  could limit its contrast and avoid amplified noise excessively in the image\cite{Zhu1999}. This feature could also be applied to globel histogram equalization, giving rise to contrast limited histogram equalization. These histogram-based techniques for image enhancement could even be stretched over two-dimensional histogram or more\cite{Celik2012}.
\subsection{Histogram-based Descriptors}
Histogram-based descriptors such as color histograms or histograms of oriented gradients are extensively used in computer vision.
\subsubsection{Some Descriptors for Histogram-based Search}
Histogram-based search is to tackle the problem of searching a template in a test image using the histogam-based representations. Here, several standard histogram-based descriptors for histogram-based search are explored as follows.\\ 
The well-known integral histogram is presented to easily and fastly compute histograms of all possible target regions in a given data, yet the computational cost and the requirement of memory for which are proportional to the number of histogram bins\cite{Porikli2005}. Another method termed distributive histogram is based on fast median filtering, which outperforms the previous approach\cite{M.SizintsevK.G.Derpanis2008}. But the computational cost and memory requirement of the distributive histogram are still proportion to the number of histogram bins. Thus, other algorithms are proposed such as a square-root sampling approach\cite{Chang2010} and min-space integral histogram of integral histogram\cite{Dubuisson2012}.
\subsubsection{HOG and Some Extensions of HOG}
Histogram of oriented gradients (HOG), a well-known feature description based histogram, counts occurrences of gradient orientation in each block of a given image. HOG outperforms almost all the other feature descriptors (e.g., SIFI, orientation histogram\cite{Freeman1995}) due to its robustness to illumination variation and invariance to the geometric and photometric transformations. Accordingly, many extensions based HOG have been proposed in the past decade such as histogram of oriented lines (HOL) for palmprint recognition\cite{Jia2014}, Co-occurrence HOG (Co-HOG) and Convolutional Co-HOG (ConvCo-HOG) for recognition of texts in scenes\cite{Tian2015}, rotation-invariant histograms of oriented gradients (Ri-HOG) for image retrieval\cite{Chen2015}, extended Histogram of Gradients (ExHoG) for human detection\cite{Satpathy2014}, etc. 
\subsubsection{Other Histogram-based Descriptors}
There are many other histogram-based descriptors such as fast point feature histograms (FPFH) for 3D registration\cite{Rusu}, Symmetric-aware Flip Invariant Sketch Histogram (SYM-FISH) to refine the shape context feature for describing a sketch image\cite{Cao2013}, multi-texton histogram (MTH) for image retrieval\cite{Liu2010}, 
Unique signatures of histograms (SHOT) for surface and texture description\cite{Salti2014}, contrast context histogram (CCH) for image matching and object recognition
\cite{Huang2008},
\subsection{Histogram with Spatial Information}
The conventional histograms are inadequate for many applications since they suffers from the inability to capture any spatial image information. An obvious way to extend this feature is to compute the histograms of multiple resolutions of an image to form a multiresolution histogram. Since multiresolution histograms combine spatial information with histogram, they are able to discriminate between different images even if the images have same histograms\cite{Hadjidemetriou2001,Hadjidemetriou2004}. Instead of the indirect use of spatial information, Spatiogram (or spatial histogram)\cite{Birchfield2005}and correlogram\cite{Huang}were presented to encode spatial information directly into histograms. However, all these extensions still have difficulties in distinguishing images, thus, adopting markov chain models to characterize the spatial co-occurrence of histogram patterns was proposed to form the so-called markov stationary features\cite{Li2008}. A extension of markov stationary features is contextualizing histogram, which could encode more complicated spatial imformation by expanding to higher order contextualized histograms\cite{Kassim2009,Feng2012}. 
\section{Multiresolution Histogram and The Meaning of Multiresolution}
A multiresolution histogram is a set of histograms of an image at multiple resolutions through applying multiresolution decomposition to the original image. Multiresolution decomposition with filters is common in practical application, especially adopting gaussian filter. In this way, the resolution of input image will be decreased by convolving the input image with a gaussian filter function and a multiresolution histogram can be easily computed by a filtered image.
\section{Multi-dimensional Histogram}
\bibliographystyle{ieee}
\bibliography{references.bib}
\end{document}
